\chapter{Methodology}

In this chapter, the methodology for implementation of the learning algorithms and its various constituents are described.
In addition to it, confidence bounds are delved into briefly.

\section{Smooth estimate of Transition Probability}

As the process of interaction of the agent with the environment is inherently noisy, a smoothing of the transition probability estimates is applied to aid learning.
The smooth probability estimate is defined as

\begin{equation}
    P_o(s, s^\prime) = \frac{P_o(s, s^\prime) + \alpha}{\sum_{x \in \mathcal{S}} \left(P_o(s, x) + \alpha \right)}
\end{equation}

where, $\alpha$ is the smoothing factor. This is implemented as

\begin{minted}[fontsize=\footnotesize, breaklines]{python3}
for s in range(self.n_states):
    # Smooth estimate.
    p_hat[s] = (p_hat[s] + alpha)/(np.sum(p_hat[s]) + alpha * self.n_states)
\end{minted}



\section{Stationary Distribution of Markov Chain}

The stationary distribution of the Markov chain describes the distribution of the states after a sufficiently long interval of time such that the distribution is not time varying.
If $\mu$ is a column vector representing the probabilities of the states that the Markov chain can visit, then $\mu$ is the stationary distribution of the Markov chain if it satisfies 

\begin{equation}
    \label{eqn:mu_estimate}
    \mu^\top = \mu^\top P
\end{equation}


where $P$ is the transition probability matrix of the Markov chain. 
In other words, $\mu$ is the fixed point of the above equation.
Transition probabilities defining ergodic Markov chains are guaranteed to have a stationary distribution of states.

From equation \ref{eqn:mu_estimate}, it can be seen that $\mu$ is the left eigenvector of $P$ with eigenvalue 1.
Hence, $\mu$ can be computed by eigendecomposition of the transition probability matrix $P$, then the eigenvector whose eigenvalue is 1 can be suitably normalized to obtain the stationary distribution.

It is implemented as follows

\begin{minted}[fontsize=\footnotesize, breaklines]{python3}
def estimate_stationary_distribution(self, transition_probability):
    eiv, r_evec = np.linalg.eig(transition_probability)
    l_evec = np.linalg.pinv(r_evec)

    # Find index of eigenvector with eigenvalue of unity.
    pick_index = np.where(np.abs(eiv - 1) <= self.eps)
    pick_index = pick_index[0]

    t_estimate = np.zeros(transition_probability.shape[0])

    for idx in pick_index:
        t_estimate += np.abs(l_evec[idx])

    # Normalize probabilities.
    estimate = t_estimate / np.sum(t_estimate)

    return estimate
\end{minted}


\section{Transition Probability}

The true transition probability induced by an option $o$, $P_o$ as described in Section \ref{sec:trueP0} is computed as follows

\begin{minted}[fontsize=\footnotesize, breaklines]{python3}
def trueP0(self, mdp):
    # terminal states
    absorbing = [(s, p) for (s, p) in self.termination_condition if p > 0.0]
    # inner: all states that are not terminal
    inner = [(s, p) for (s, p) in self.termination_condition if p < 1.0]
    # internal: internal states with initial state removed
    internal = [(s, p) for (s, p) in inner if s not in self.initiation_states]

    # compute true matrix P0
    p0 = np.zeros((len(inner) + len(absorbing), len(inner) + len(absorbing)))
    q0 = np.zeros((len(inner), len(inner)))
    v0 = np.zeros((len(inner), len(absorbing)))

    for i, (s, _) in enumerate(inner):
        for j, (s_prime, p_s_prime) in enumerate(inner):
            q0[i][j] = (1 - p_s_prime) * mdp.transition_probabilities[self.o_index][s][s_prime]

    for i, (s, _) in enumerate(inner):
        for k, (s_prime, p_s_prime) in enumerate(absorbing):
            v0[i][j] = p_s_prime * mdp.transition_probabilities[self.o_index][s][s_prime]

    p0[:len(inner), :len(inner)] = q0
    p0[:len(inner), len(inner):] = v0
    p0[len(inner):, len(inner):] = np.eye(len(absorbing), len(absorbing))

    states = []
    states.extend(map(lambda el: int(el[0]), inner))
    states.extend(map(lambda el: int(el[0]), absorbing))

    return p0, states
\end{minted}

\section{Computing Optimal Policy}

The UCRL algorithm along with extended value iteration followed by computing the optimal policy is implemented as follows

\begin{minted}[fontsize=\footnotesize, breaklines]{python3}
def computepolicy(self, MDP: SemiMDP, r_upper, p_lower, p_hat, p_upper, t):
    v_old = np.zeros(MDP.n_states)
    v_diff = np.arange(MDP.n_states)
    pi = np.zeros(MDP.n_states)
    iter = 0

    while iter < 1000 and max(v_diff) - min(v_diff) > 1/np.sqrt(t):
        iter += 1

        q = self.normalizeprobs(p_lower, p_hat, p_upper, v_old)
        v = np.zeros(MDP.n_states)

        for s in range(MDP.n_states):
            v_max_over_options = 0
            maximizing_option = None

            # Max over options.
            for o in MDP.get_options(s):
                v_tmp = 0
                mu_hat = MDP.options[o].estimate_stationary_distribution(q[o])

                for s_prime in range(MDP.n_states):
                    v_tmp += r_upper[s_prime] * mu_hat[s_prime]

                v_tmp += mu_hat[s] * (np.dot(q[o][s].T, v_old) - v_old[s]) + v_old[s]

                if v_tmp >= v_max_over_options:
                    v_max_over_options = v_tmp
                    maximizing_option = o

            v[s] = v_max_over_options
            pi[s] = maximizing_option

        v_diff = v - v_old
        v_old = v

    return pi
\end{minted}

\section{Confidence Bounds}

The confidence bounds for $\mu_o$ may be calculated either implicitly or explicitly.
Fruit et al. \cite{fruit_regret_2017} use Hoeffding and empirical Bernstein type confidence intervals for the SMDP transition probabilities, the Markov chain associated with each option and the rewards.
In this work, we have used empirical Bernstein type confidence intervals to obtain upper bounds on the estimates.


\section{Putting it all together}

Experimentation is carried out with the following driver script. 
At the start of each experiment, a new environment is instantiated and the various statistics are reset.
A new policy is evaluated only once experience has doubled for at least one of the state-option pairs.

\begin{minted}[fontsize=\footnotesize, breaklines]{python3}
import numpy as np
from copy import deepcopy as dc
from tqdm.auto import trange, tqdm

from GridWorldOptions import GridWorldOptions
from ExtendedValueIteration import EVI
from util import plot_option_visitation_counts, plot_policy, plot_accumulated_reward, plot_accumulated_regret

MAX_ITERS = 200_000
NUM_EXP = 5
GRID_DIM = 5
T_MAX = 2

# Keeps track of the accumulated regret across experiments.
data = list([] for _ in range(NUM_EXP))

overall_visitation_counts = {"state": [], "option": []}
accumulated_regrets = []

for exp in trange(NUM_EXP):
    env = GridWorldOptions(GRID_DIM, r_max=1, teleport_on_reaching_goal=True, m=T_MAX)
    evi = EVI()


    pi = [0] * env.n_states
    state = env.state
    previous_experience = dc(env.N)

    true_reward = env.compute_true_reward()

    tqdm.write("Computing optimal policy...", end="")
    optimal_policy, asymptotic_average_reward = env.solve(true_reward)
    tqdm.write(f"Done! {asymptotic_average_reward = :.5f}")
    plot_policy(optimal_policy, env, f"Optimal policy in {GRID_DIM}x{GRID_DIM} grid with target {env.target}", show=False, save=True)

    accumulated_reward = 0
    accumulated_holding_time = 0
    accumulated_regret = []

    for t in trange(MAX_ITERS):
        valid_options = env.get_options(state)
        option = pi[state]

        if option not in valid_options:
            option = np.random.choice(valid_options)

        option = int(option)
        opt = env.options[option]

        prev_state = dc(env.state)
        state, reward, holding_time = env.act(option)

        overall_visitation_counts["state"].append(prev_state)
        overall_visitation_counts["option"].append(opt.name)

        accumulated_reward += reward
        accumulated_holding_time += holding_time

        accumulated_regret.append(accumulated_holding_time * asymptotic_average_reward - accumulated_reward)

        data[exp].append(accumulated_reward)

        if env.has_experience_doubled(previous_experience) or t == MAX_ITERS - 1:
            # It doesn't matter which option's method is called as the counts are used from the SemiMDP.
            _, r_ub = env.options[0].estimate_reward_bound(MDP=env)

            p_hat = env.estimate_transition_probability()
            p_lower, p_upper = env.estimate_transition_bound()

            pi = evi.computepolicy(env, r_ub, p_lower, p_hat, p_upper, t+1)
            previous_experience = dc(env.N)

    accumulated_regrets.append(accumulated_regret)
    plot_policy(pi, env, f"Learned policy in {GRID_DIM}x{GRID_DIM} grid with target {env.target}", show=False, save=True)


plot_accumulated_regret(accumulated_regrets, f"Accumulated regret in {GRID_DIM}x{GRID_DIM} grid ({NUM_EXP} experiments)", show=True, save=True)
plot_accumulated_reward(data, f"Accumulated reward in {GRID_DIM}x{GRID_DIM} grid ({NUM_EXP} experiments)", save=True)
plot_option_visitation_counts(overall_visitation_counts, f"Visitation counts (teleport) in {GRID_DIM}x{GRID_DIM} grid ({NUM_EXP} experiments)", GRID_DIM * GRID_DIM, save=True)
\end{minted}

Post completion of the simulations, the plots for accumulated regret as well as accumulated reward are generated.
The experimentation performed as well as the results obtained are described in the following chapter.
