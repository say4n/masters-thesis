\chapter{Methodology}

In this chapter, the methodology for implementation of the learning algorithms and its various constituents are described.
In addition to it, confidence bounds are delved into briefly.

\section{Smooth estimate of Transition Probability}

As the process of interaction of the agent with the environment is inherently noisy, a smoothing of the transition probability estimates is applied to aid learning.
The smooth probability estimate is defined as

\begin{equation}
    P_o(s, s^\prime) = \frac{P_o(s, s^\prime) + \alpha}{\sum_{x \in \mathcal{S}} \left(P_o(s, x) + \alpha \right)}
\end{equation}

where, $\alpha$ is the smoothing factor.

\section{Stationary Distribution of Markov Chain}

The stationary distribution of the Markov chain describes the distribution of the states after a sufficiently long interval of time such that the distribution is not time varying.
If $\mu$ is a column vector representing the probabilities of the states that the Markov chain can visit, then $\mu$ is the stationary distribution of the Markov chain if it satisfies 

\begin{equation}
    \label{eqn:mu_estimate}
    \mu^\top = \mu^\top P
\end{equation}


where $P$ is the transition probability matrix of the Markov chain. 
In other words, $\mu$ is the fixed point of the above equation.
Transition probabilities defining ergodic Markov chains are guaranteed to have a stationary distribution of states.

From equation \ref{eqn:mu_estimate}, it can be seen that $\mu$ is the left eigenvector of $P$ with eigenvalue 1.
Thus, $\mu$ can be computed by eigendecomposition of the transition probability matrix $P$. 
The eigenvector whose eigenvalue is 1 can be suitably normalized to obtain the stationary distribution of the Markov chain.

\section{Representing Options as absorbing Markov chains} \label{sec:trueP0}

Since the options terminate in a finite time number of time steps (from assumption \ref{asm:option}), options can be equivalently represented as Markov reward processes whose state space is defined as the set of all states that a reachable from the option and whose terminal states are the absorbing states of the Markov chain.
Formally, for an option $o$, the inner set of states $\mathcal{S}_o$ is defined as the set of states $s$ which are reachable from the initiation state $s_o$ by executing the option policy $\pi_o$ with $\beta(s) < 1$ along with the initiation state $s_o$, while the set of absorbing states is defined as $\mathcal{S}_o^{abs}\{s : \beta(s) > 0\}$.
The absorbing Markov chain associated with an option $o$ is defined by a transition probability matrix $P_o$ of shape $(|\mathcal{S}_o| + |\mathcal{S}_o^{abs}|) \times (|\mathcal{S}_o| + |\mathcal{S}_o^{abs}|)$ described below

\begin{equation}
    P_o = \begin{bmatrix}
       Q_o & V_o \\
       0 & I
    \end{bmatrix}
\end{equation}

where, $Q_o(s, s^\prime)$ is the transition matrix between inner states of dimension $|\mathcal{S}_o| \times |\mathcal{S}_o|$

\begin{equation}
    Q_o(s, s^\prime) = (1 - \beta_o(s^\prime)) p(s^\prime \mid s, \pi_o(s)) \forall s, s^\prime \in \mathcal{S}_o
\end{equation}

$V_o(s, s^\prime)$ is the transition matrix from inner states to absorbing states of dimension $|\mathcal{S}_o| \times |\mathcal{S}_o^{abs}|$

\begin{equation}
    V_o(s, s^\prime) = \beta_o(s^\prime) p(s^\prime \mid s, \pi_o(s)) \forall s \in \mathcal{S}_o, s^\prime \in \mathcal{S}_o^{abs}
\end{equation}

and $I$ is an identity matrix of dimension $|\mathcal{S}_o^{abs}| \times |\mathcal{S}_o^{abs}|$.
The rest of the elements in the transition probability matrix are zeros.

As the expected cumulative reward $\bar{R}(s, o)$ and the duration of options $\bar{\tau}(s, o)$ are directly related to $Q_o$ and $V_o$ of the associated transition probability matrix $P_o$ for an option, an estimate of this transition probability matrix can be exploited to derive corresponding estimates for the expected cumulative reward and the duration of options.
Similarly confidence intervals on these could also be propagated from the confidence intervals of the corresponding values of the transition probability matrix.
However, this does not work very well as if the the value 0 belongs to the confidence interval of $P_0$, then the optimistic estimates of the expected cumulative reward $\bar{R}(s, o)$ and the duration of options $\bar{\tau}(s, o)$ are not bounded and ill defined as $\bar{\tau}(s, o)$ appears in the denominator of equation \ref{eq:optimal_bias}.


\section{Representing Options as irreducible Markov chains}

Starting from $P_o$, an irreducible Markov chain can be constructed whose stationary distribution relates to the ratio of the cumulative reward and the holding times as well as the inverse of the holding times.
All the terminal states are merged together and are redirected to the initiation state of the option.

The optimistic optimality equation is defined by Fruit et al. \cite{fruit_regret_2017} in equation 5 as

\begin{equation}
    \label{eqn:fruit5}
    \tilde{\rho}^* = \max_{o \in \mathcal{S}} \left\{ \max_{\tilde{\mu}_o, \tilde{r}_o} \left\{ \sum_{s^\prime \in \mathcal{S}_o} \tilde{r}_o(s^\prime) \tilde{\mu}_o(s^\prime) + \tilde{\mu}_o(s) \left( \max_{\tilde{b}_o}\left\{ \tilde{b}_o^\top \tilde{u}^* \right\} - \tilde{u}^*(s) \right)  \right\} \right\}
\end{equation}

where, $ \tilde{r}_o(s^\prime) =  \tilde{r}(s^\prime, \pi_o(s^\prime))$ is the optimistic reward, $\tilde{b}_o = (\tilde{p}(s^\prime \mid s, o))_{s^\prime \in \mathcal{S}_o}$ is the optimistic transition probability vector and $\tilde{\mu}_o$ is the stationary distribution of the estimated irreducible Markov chain.

To obtain an approximation of equation \ref{eqn:fruit5}, starting from an estimate $u_0(s) = 0$ Fruit et al. define an iterative update rule described as follows

\begin{equation}
    \label{eqn:fruit7}
    u_{j+1}(s) = \max_{o \in \mathcal{S}} \left\{ \max_{\tilde{\mu}_o} \left\{ \sum_{s^\prime \in \mathcal{S}_o} \tilde{r}_o(s^\prime) \tilde{\mu}_o(s^\prime) + \tilde{\mu}_o(s) \left( \max_{\tilde{b}_o}\left\{ \tilde{b}_o^\top u_j(s) \right\} - u_j(s) \right) \right\} \right\} + u_j(s)
\end{equation}

where, $u_j(s)$ is the value of state $s$ at the end of the $j$-th time-step.


\section{Confidence Bounds}

Fruit et al. \cite{fruit_regret_2017} use Hoeffding and empirical Bernstein type confidence intervals for the SMDP transition probabilities $\mathcal{P}$, the Markov chain associated with each option characterised by their transition probabilities $P_o$ and the rewards realized $r$.
In this work, we have used empirical Bernstein type confidence intervals to obtain upper bounds on the estimates.


\section{Putting it all together}

At the start of each experiment, a new environment is instantiated and the various statistics are reset.
A new policy is evaluated only once experience has doubled for at least one of the state-option pairs.
Post completion of the simulations, the plots for accumulated regret as well as accumulated reward are generated.
The experimentation performed as well as the results obtained are described in the following chapter.
