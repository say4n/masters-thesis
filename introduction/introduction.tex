\normallinespacing

\chapter{Introduction}

Markov Decision Processes (MDPs) are mathematical models with a finite set of states $\mathcal{S}$ and a finite set of actions $\mathcal{A}$ wherein an agent is tasked with taking an action $a$ in state $s$.
Upon executing the action, the agent may transition to another state $s^\prime$ and receives a random reward from the environment, typically bounded in the range $[0, 1]$.

Reinforcement learning is a paradigm of learning for MDPs that incorporates learning from delayed feedback.
Typically the performance of reinforcement learning algorithms is evaluated based on the performance of a learned policy \cite{sutton_reinforcement_2018, kearns_finite-sample_1999}.
However, algorithms like UCRL2 \cite{jaksch_near-optimal_2010} by Jaksch et al. evaluate the performance of a learning algorithm during the learning process by comparing its performance with that of the optimal policy for a particular setting.
To that end, the accumulated reward by following the learned policy is compared to the accumulated reward while executing actions according to the optimal policy.
The difference between these two quantities is defined as the regret of the learning algorithm with respect to the optimal policy.

Efficient learning over large time horizons in complex environments requires hierarchical reasoning.
Temporally extended actions is one such abstraction.
MDPs in their conventional form do no involve actions that are temporally extended.
This renders conventional MDPs incapable of exploiting the efficiencies offered by a higher level of abstraction.
To ameliorate this, Sutton et al. proposed the options framework \cite{sutton_between_1999} which is an extension to the theory of semi-Markov Decision Processes (SMDPs) introduced by Parr \cite{parr_reinforcement_nodate}, Bradtke and Duff \cite{bradtke_reinforcement_1995} and Mahadevan et al. \cite{mahadevan_self-improving_1997}.
SMDPs are a special type of MDP wherein actions take variable amounts of time and can model temporally extended course of actions.
Temporal abstraction intuitively leads to faster learning as it helps learn promising sequences of actions instead of a single action.
It also decreases the computational overhead as the action space can be vastly reduced by introducing options thereby reducing the exploration necessary to evaluate the actions.
However, introducing options may lead to gains in performance in limited contexts as shown by Jong et al. in \cite{jong_utility_2008}.
This necessitated a formal analysis on the performance of options and when it aids reinforcement learning, which was undertaken by Fruit et al. in \cite{fruit_exploration--exploitation_2017}. 
The precursor to this work was Brunskill and Li's \cite{brunskill_pac-inspired_2014} derivation of sample complexity bounds for R\textsc{max} like algorithms for SMDPs and Fruit and Lazaric's \cite{fruit_exploration--exploitation_2017} mapping of regret of the SMDP version of UCRL algorithm to the regret of learning in the original MDP with options.
Fruit et al. \cite{fruit_exploration--exploitation_2017} derive upper bounds for regret in the SMDP setting.

In this work, we implement the work by Fruit et al.\cite{fruit_exploration--exploitation_2017} and try various confidence bounds like the empirical Bernstein peeling confidence bound to gauge the accumulated regret.

This thesis is structured into chapters that delve briefly into the theory of reinforcement learning, then the options framework and semi-Markov decision processes.
This is followed by a chapter dedicated to the learning algorithms followed by our methodology.
The two final chapters describe our experimental setup, the experimentation performed, our results, and the conclusion.


\newpage
